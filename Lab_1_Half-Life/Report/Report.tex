%\documentclass[12pt,letterpaper,onecolumn]{article}
%\documentclass[11pt,letterpaper,onecolumn]{article}
\documentclass[10pt,letterpaper,onecolumn]{article}
%\documentclass[12pt,letterpaper,twocolumn]{article}
%\documentclass[11pt,letterpaper,twocolumn]{article}
%\documentclass[10pt,letterpaper,twocolumn]{article}


%\usepackage{amsmath}
%\usepackage{graphics}
\usepackage{graphicx} %more modern version of graphics
%\graphicspath{{path-to-folder-containing-necessary-graphics}{other folder as necessary}}


%=====================================================
%============ \begin{document} =======================
%=====================================================

\begin{document}

%=====================================================
%============ Title ==================================
%=====================================================

\title{\Large\bf Measuring the Half Life of Ba-137 Using a Scintillator Detector}
% \title{\Large\bf Larger, Bolded Title}

%=====================================================
%============ Author =================================
%=====================================================
\author{
 Akhil Deshpande \\*
  \\*
 PHY 353L Modern Laboratory \\*
 Department of Physics \\*
 The University of Texas at Austin \\*
 Austin, TX 78712, USA
}
\date{September 15, 2023}

% \address{The University of Texas, Austin, Texas, 78712}

\maketitle

%=====================================================
%============ Abstract ===============================
%=====================================================

\begin{abstract}

In an experiment designed to study very interesting phenomena
we have observed fascinating events.
This result confirms our understanding of the underlying physics.
Few statements summarizing the technique and results.

\end{abstract}

%=====================================================
%============ Body of the article ====================
%=====================================================

%=====================================================
%============ Section ================================

\section{Introduction}

\subsection{Physics Motivation}

Broad physics motivation should be discussed briefly but
meaningfully. Basic phenomena should be
explained (or referred to) and
prediction for experimental results clearly
stated. Here and throughout the report appropriate
references should be included~\cite{book, article}.

\subsection{Historical context}

You may want to relate what you are doing to first or previous
work on this topic. Since you are doing an experimental work,
the context should be on the experimental technique. For example,
you may say that this was first done in a such and such way
but later it was discovered
that one can also do it another way. Your technique may be
related to the first or none of the above.


%=====================================================
%============ Section ================================

\section{Theoretical background}

Provide some more theoretical details for your measurements.
Give formulas and references which provide a specific theoretical
context for your measurements.


%=====================================================
%============ Section ================================

\section{Experimental setup}


\subsection{Apparatus}

Ideas behind the particular technique should be briefly
discussed. Enclose references. Sketches, pictures, and
suitable schematics should be included and explained
concisely. All major components of the system should be
mentioned and their role clearly motivated. This section
is not simply a list of components and it is not an
instruction manual. 


%=====================================================
%============ Importing pictures  ====================
%=====================================================

\begin{figure}[h]
  %
  % placement specifier = { h,t,b,p,!,H }
  % see the following url for placement specifier definitions:
  % http://en.wikibooks.org/wiki/LaTeX/Floats,_Figures_and_Captions
  %
 \begin{center}
%  \includegraphics*[width=3.5in]{string_theory.pdf}
  %
 \caption{ My Caption, in all its glory.\label{fig:apparatus} }
 % See http://en.wikibooks.org/wiki/LaTeX/Labels_and_Cross-referencing
 %  for information on labels.
 \end{center}
\end{figure}

\subsection{Data Collection}

Data taking procedures should be described and various modes of
data collection explained. Calibration procedures and
relevant plots and numerical tables should be included.
State clearly what measurements were taken for the final
data analysis. Describe `doing the experiment' so it would
be helpful to other students in the future. This may need
to include physics arguments {\em what } and {\em how } data should
be collected.


\subsection{Data Analysis}

This is the most important section of the report.
Describe data analysis. Details! Perhaps include a figure and refer
the reader to it! See Figure~\ref{fig:apparatus}. Maybe you will need
to include a table. See Table~\ref{tab:events}.

Describe calculations of the final results.
Thoroughly address error analysis and discussion of measurement
uncertainties. Remember: NO EXPERIMENTAL RESULT CAN BE QUOTED
WITHOUT AN ERROR BAR! Do not forget about random or systematic
uncertainties. Be sure to propagate errors correctly!
Include a demonstrative graph when possible.
%See Figure~\ref{fig:results}.


Make final assessment and interpretation after that.
Discuss apparatus problems if any. Suggestions for
lab setup or approach improvements are welcome!

%=====================================================
%============ Importing pictures  ====================
%=====================================================

\begin{figure}[h]
  %
  % placement specifier = { h,t,b,p,!,H }
  % see the following url for placement specifier definitions:
  % http://en.wikibooks.org/wiki/LaTeX/Floats,_Figures_and_Captions
  %
 \begin{center}
%  \includegraphics*[width=3.5in]{centrifugal_force.pdf}
  %
  %
 %\caption{ The experiment provided interesting results.\label{fig:results} }
 % See http://en.wikibooks.org/wiki/LaTeX/Labels_and_Cross-referencing
 %  for information on labels.
 \end{center}
\end{figure}


%===========================================================================
%=========================== Table 1 =======================================
%===========================================================================
%
% Note: the position of the table does not always depend on its position here. See
% http://en.wikibooks.org/wiki/LaTeX/Tables
% for details.
%

\begin {table}[h]
{
{%\footnotesize
\begin {center}
\begin {tabular} {c | c c  c | c | c c }
\hline\hline
Run 			&   ~~POT~~ 		&
\multicolumn{2}{ | c } {Predicted}  &   \multicolumn{2}  {| c} {Selected} \\
Period		& $(10^{20})$	&
\multicolumn{2}{ | c } {(No oscillations)}  &   \multicolumn{2}  {| c} {(Far Detector)} \\
			    &
			& \multicolumn{1} {| c } {~~~Fully} & \multicolumn{1} { c } {~~~Partially}
			& \multicolumn{1} {| c } {~~~Fully} & \multicolumn{1} { c } {~~~Partially} \\
			
\hline
I			& 1.269		
			& \multicolumn{1} {| r } {426 } & \multicolumn{1} { r } {375 }
		     	& \multicolumn{1} {| r } {318 } & \multicolumn{1} { r } {357 } \\

II		     	& 1.943
			& \multicolumn{1} {| r } {639 } & \multicolumn{1} { r } {565 }
		    	& \multicolumn{1} {| r } {511 } & \multicolumn{1} { r } {555 } \\

\hline
Total			& 7.246
			& \multicolumn{1} {| r } {2,451 } & \multicolumn{1} { r } {2,206 }
		     	& \multicolumn{1} {| r } {1,986 } & \multicolumn{1} { r } {2,017 } \\

\hline% \hline
\end {tabular}
\end {center}
}
}
\caption {\label{tab:events}
Predicted and observed numbers of events classified in the Far Detector as fully and
partially reconstructed charged current interactions shown for all running periods.
 }
\end {table}


%=====================================================
%============ Section ================================
%=====================================================

\section{Results}

Clearly present the result of your analysis. Make sure
you include the uncertainties. No experimental result
can be quoted without an error attached to it.

Your results should be compared with predictions and other
measurements.


%=====================================================
%============ Section ================================

\section{Summary and conlcusions}

Summarize briefly the results of the experiment.
Acknowledge (i.e., thank for) contributions or help
of your partner(s) and or
others (TA, machine shop, software used, ...).

%=====================================================
%============ Bibliography  ==========================
%=====================================================

\begin{thebibliography}{9}

\bibitem{book} R. Feynman, {\it QED}, Ch.7.

\bibitem{article}	
R.Dalitz, Proc. Roy. Soc. (London) {\bf A64}, 667 (1951)

\end{thebibliography}

%=====================================================
%============ End ====================================
%=====================================================

\end{document}

%=====================================================
%============ End ====================================
%=====================================================
